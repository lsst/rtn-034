\documentclass[DM,authoryear,toc]{lsstdoc}
% lsstdoc documentation: https://lsst-texmf.lsst.io/lsstdoc.html
\input{meta}

% Package imports go here.

% Local commands go here.

%If you want glossaries
%\input{aglossary.tex}
%\makeglossaries

\title{Planning Tools for Rubin Operations}

% Optional subtitle
% \setDocSubtitle{A subtitle}

\author{%
Phil Marshall, 
Amanda Bauer,
Bob Blum,
Cathy Petry,
Christine Soldahl
}

\setDocRef{RTN-034}
\setDocUpstreamLocation{\url{https://github.com/lsst/rtn-034}}

\date{\vcsDate}

% Optional: name of the document's curator
% \setDocCurator{The Curator of this Document}

\setDocAbstract{%
This document describes the system for planning budgets and staffing in Vera C. Rubin Observatory Operations. The current implementation is a set of custom, inter-dependent, collaborative workbooks that capture the flow of information from work breakdown structure (WBS) through to 1) labor and non-labor cost estimation, and 2) staffing. This document describes the design, recommended operation, and maintenance of these planning tools.
}

% Change history defined here.
% Order: oldest first.
% Fields: VERSION, DATE, DESCRIPTION, OWNER NAME.
% See LPM-51 for version number policy.
\setDocChangeRecord{%
  \addtohist{1}{2022-04-28}{Initial version.}{Phil Marshall}
}


\begin{document}

% Create the title page.
\maketitle
% Frequently for a technote we do not want a title page  uncomment this to remove the title page and changelog.
% use \mkshorttitle to remove the extra pages

% ADD CONTENT HERE
% You can also use the \input command to include several content files.

\section{Requirements}
\label{sec:reqs}

\section{Design}
\label{sec:dsgn}

\section{Implementation}
\label{sec:impl}

\section{Operation}
\label{sec:oper}

\section{Maintenance}
\label{sec:mntn}



\appendix
% Include all the relevant bib files.
% https://lsst-texmf.lsst.io/lsstdoc.html#bibliographies
\section{References} \label{sec:bib}
\renewcommand{\refname}{} % Suppress default Bibliography section
\bibliography{local,lsst,lsst-dm,refs_ads,refs,books}

% Make sure lsst-texmf/bin/generateAcronyms.py is in your path
\section{Acronyms} \label{sec:acronyms}
\input{acronyms.tex}
% If you want glossary uncomment below -- comment out the two lines above
%\printglossaries





\end{document}
